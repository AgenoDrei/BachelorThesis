\subsection{Hardware}

Unabhängig von der Zusammenstellung umfänglicher Speicherlösungen, müssen die Daten trotzdem irgendwann auf einen Hardwarespeicher geschrieben werden. Deswegen ist es notwendig die Grundlagen zu kennen, um die Herausforderungen komplizierter Lösungen zu verstehen. Folgender Abschnitt erläutert die allgemeine Funktion und geht auf einige der am häufigsten verwendeten Typen ein.

\subsubsection{Hard Disk Drive (HDD)}

Hierbei handelt es sich um einen nicht volatilen Speicher, der digital kodierte Inhalte auf der magnetischen Oberfläche einer Platte speichert. Traditionell gibt es zwei Protokolle: \ac{SAS} im kommerziellen und \ac{SATA} im privaten Sektor \parencite{wikibooks.2016}.

Ein oder mehrere Leseköpfe werden mithilfe von Armen und einem Servomotor über die Platte bewegt, um relevante Stellen auszulesen. Bei der Herstellung von HDDs werden Tracks in die Platte geschrieben, die jeweils ein Label und Blockinformationen besitzen. Durch sie ist es einfacher, den Lesekopf zu platzieren. Jeder dieser Tracks kann bis zu einigen MBs an Daten beinhalten. Durch die Blockbildung bei I/O-Operationen kann häufig nur ein einzelner Block (64-256 Byte) geschrieben werden, wodurch Festplatten unnötig langsam werden. Durch neue Technologie können IOs in eine Warteschlange eingereiht werden, sodass die Platte kontinuierlich lesen und schreiben kann. Trotzdem ist ein maximaler Durchsatz von ungefähr 300 \gls{IOPS} das Limit für Platten mit 10.000 RPM \parencite[Kap. 3]{kaufmann.2016}.

Jede Festplatte hat heutzutage einige private Sektoren, auf denen Reparaturinformationen und Reserveblöcke gespeichert sind. Fällt ein Block in der Festplatte aus (durch zum Beispiel physikalische Beschädigung), kann dieser fließend ersetzt werden, sodass keine Daten verloren gehen. Kurz vor dem Ausfall einer Platte steigt diese Anzahl der Lesefehler einer Platte massiv an, sodass eine Datenreovery versucht werden kann. Leider ist dies keine verlässliche Methode, um Daten zu sichern, sodass andere Methoden verwendet werden sollten \parencite[Kap. 3]{kaufmann.2016}.

Um Datenverlust durch Plattenausfälle entgegenzuwirken, besteht die Möglichkeit der Verwendung von so genannten \acs{RAID}. Dies ist eine Methode, bei der mehrere HDDs so zusammengeschaltet werden, sodass bei dem Ausfall einer Einzelnen die Daten weiterhin verfügbar sind \parencite{wikibooks.2016}.

Festplatten werden seit Jahrzehnten verwendet und sind gut erforscht, es treten extrem selten unbekannte Probleme auf, ihr Speicher ist extrem günstig geworden und in Masse verfügbar. Ihr größter Nachteil ist die sehr geringe I/O-Geschwindigkeit im Vergleich mit neueren Technologien \parencite[Kap. 3]{kaufmann.2016}.

\subsubsection{Solid State Drive (SSD)}

Im Jahr 2007 wurde der erste \acs{SSD}-Speicher vermarktet und stach durch extrem hohe Zugriffsraten heraus. Diese waren bei 4000 \gls{IOPS} bereits um das Zehnfache höher als bei klassischer Festplatten. Heutzutage ist es möglich IOPS Zahlen im Millionenbereich zu erreichen, was eine Revolution für Speicher bedeutet \parencite[Kap. 3]{kaufmann.2016}.

\begin{figure}[hbt]
	\centering
	\includegraphics[scale=0.85]{images/flash}
	\caption{Funktionsweise eines NAND-Flashspeichers \parencite{kaufmann.2016}}
	\label{fig:flash}
\end{figure}

\acsp{SSD} benutzen Nicht-Und (NAND) memory chips zum Speichern einzelner Bitinformationen. Von diesen Zellen gibt es dann einige Millionen, die beliebig adressiert werden können (Random Access Memory). Innerhalb eines so genannten Floating Gate können nun Elektronen eingefangen werden, um eine binäre Eins zu repräsentieren. Im Ausgangszustand sind diese ``Fallen'' leer, um Nullinformationen zu speichern. Sobald der Speichercontroller eine Zelle laden will, wird der Einlasstransistor geöffnet und einige Elektronen in der Zelle für eine beliebig lange Zeit gefangen.

Dieser Prozess findet in \autoref{fig:flash} statt, wenn auf der Bit- und Wordline eine positive Ladung gesetzt wird. Meistens liegen mehrere Zellen auf einer Reihe, was den Leseprozess verkomplizieren kann. Es wird eine niedrige Spannung an alle Wordlines angelegt, wodurch bei geladenen Zellen eine Konduktivität festgestellt werden kann. 

Wird eine einzelne Zelle zu häufig beschrieben, kann diese Isolation verlieren, wodurch sie unnutzbar zum Speichern wird. Schreibstrategien und Reservezellen werden verwendet, um dem vorzubeugen.	

Zum Löschen der Zellen wird eine hohe negative Spannung an die Zellen angelegt, wodurch die gefangenen Elektronen aus den Floating Gates heraus gestoßen werden. Dies passiert meistens in fest definierten Pages, sodass keine speziellen Blöcke gelöscht werden können. Nach dem Löschen eines Blockes wird dessen Zeiger auf eine vorher geleerte Page bewegt und der Block als ``discarded'' markiert. Sind innerhalb einer Page genug Blöcke in diesem Zustand, werden die verbleibenden Daten verschoben und die gesamte Page gelöscht. Sie wird dann zum Pool des freien Speichers der SSD hinzugefügt \parencite{kaufmann.2016}. 
Durch diesen Prozess ist das Löschen von Dateien nicht mehr zuverlässig, da die Daten noch für lange Zeit innerhalb eines ``discarded'' Blockes liegen können, der erst wesentlich später vom Controller gelöscht wird.

SSDs können über das SATA-Protokoll angeschlossen werden, aber in den letzten Jahren wurde das wesentlich performantere NVMe entwickelt.

Da dieser Speicher keine sich bewegenden Teile hat, ist er wesentlich robuster was Schläge, Vibrationen und hohe Temperaturen angeht. Ebenfalls ist der Strombedarf wesentlich geringer, was zusammen mit den vorherigen Punkten den Betrieb von SSDs in großen Rechenzentren wesentlich einfacher und günstiger macht.

Trotzdem gibt es auch einige Nachteile von Flashspeicher: Die meisten Anwendungen sind nicht darauf ausgelegt, so hohe Zugriffszahlen zu unterstützen. In der Vergangenheit hat I/O immer ein Bottleneck in Applikationen dargestellt, sodass die meisten Programme entsprechend designed wurden.
Dies zieht sich durch die gesamte Softwarelandschaft und auch die geläufigen Betriebssysteme sind nicht in der Lage, diese neue Geschwindigkeit voll auszunutzen. Ebenfalls ist das SATA Interface nicht perfekt für SSDs, da es ursprünglich dafür entwickelt wurde, die niedrigen Bandbreiten von Festplatten auszugleichen \parencite[Kap. 3]{kaufmann.2016}.


\subsubsection{Bandlaufwerk}

Bandlaufwerke funktionieren im Grunde wie die früher verwendeten Kassetten. Innerhalb des Laufwerkes befindet sich ein dünner Streifen aus Plastik mit einer magnetischen Oberfläche. Dieser wird - ähnlich wie bei Festplatten - mit digitalen Daten beschrieben. 

Es kann nicht auf beliebige Bereiche des Bandes schreibend zugegriffen werden. Lesen und Schreiben erfolgt immer sequentiell \parencite{adrc.2009}.

Ein Bandlaufwerk verwendet einen kleinen Motor, um das Band auf- und abzuwickeln. Dabei wandert dieses entlang Lese- und Schreibköpfen. Um Unterschiede zwischen der Geschwindigkeit der ankommenden Daten vom Computer und der begrenzten \gls{IOPS} des Laufwerkes auszugleichen, wird eine Steuereinheit verwendet. Diese regelt Fehlerhandling, Puffer und andere logische Operationen.

Informationen werden in die Steuereinheit geladen und dann auf das Band geschrieben. Dieser Prozess wiederholt sich solange, bis keine Daten mehr im Puffer vorhanden sind.

Aufgrund von geringen Kosten und hoher Lebenszeit wird diese Art von Laufwerk immer noch häufig verwendet. Besonders oft wird es für Backupszenarien genutzt, da hier die Nachteile des nur sequentiellen Zugriffes auf Daten keine große Rolle spielen \parencite{adrc.2009}. 

\subsection{Speicheranbindung über Netzwerke}

An dieser Stelle werden einige populäre Technologien untersucht, um großen Mengen von Hardwarespeichern in Netzwerken oder Rechenzentren zusammenzuschließen. Dies ist notwendig, da teilweise gigantische Speichermengen parallel von verschiedenen Servern erreicht, verändert und gespeichert werden müssen.

\subsubsection{Direct Attached Storage (DAS)}

Bei \ac{DAS} handelt es sich um externen Speicher, der direkt mit einem oder mehreren Servern über beispielsweise ein \gls{SCSI} Interface verbunden ist. Dabei wird kein Netzwerk verwendet. Es gibt zwei verschiedene Typen von \ac{DAS}, bei der ersten Art werden beliebige Platten zusammen- (\ac{JBOD}) und bei der zweiten sogenannten \acs{RAID} (Arrays von Festplatten, die vor Speicherverlust schützen) angeschlossen.
Hierbei handelt sich um den ältesten Speichertypus, der lange vor dem Entstehen von SAN oder NAS verwendet wurde.

Es kann nur eine begrenzte Zahl von \ac{DAS} an einen Rechner angeschlossen werden, da die Anzahl von Schnittstellen an Servern limitiert ist. Dadurch ist diese Art von Speicher leider nur begrenzt skalierbar.

Ein weiterer großer Nachteil von direkt angebundenen Speicher ist, dass bei einem Ausfall des Servers ein wesentlicher Teil der Daten nicht mehr erreichbar ist. Die Verfügbarkeit der Daten hängt also nicht nur vom Speicher selbst ab, sondern auch von dem bereitstellenden Server.

Wird der Speicher an mehrere Server angeschlossen, um obigen, Problem entgegenzuwirken, erhöht sich durch einen Ausfall die Zugriffszeit beträchtlich \parencite[Kap. 1, Disk Storage Systems]{gupta.2002}.

Der größte Vorteil von DAS Geräten ist, dass sie einfach aufzusetzen und zu warten sind, da keine Netzwerkkenntnisse oder -hardware benötigt werden. Dies bedeutet geringe Kosten und einfache Handhabung \parencite{beal.2017}.

\subsubsection{Network Attached Storage (NAS)}

\ac{NAS} ist ein einfaches System, um Daten an einer einheitlichen Stelle im Netzwerk zu speichern. Hierdurch werden die Daten von den Servern selbst zu speziell zur Speicherung ausgelegten Hardware wegbewegt. 

Ein NAS-Gerät ist häufig spezialisierte Hardware mit meist zugehöriger Management Software. In den meisten Fällen muss es nur mit dem Firmennetzwerk verbunden und eingeschaltet werden. Für großen Firmen gibt es auch größere kompliziertere Setups.
Ebenfalls ist der Speicher plattformunabhängig und kann mithilfe verschiedener Protokolle von jedem Client verwendet werden. Einige Beispiele für diese sind: HTTP (Internetanfragen), FTP oder TCP/IP. Ebenfalls können Datenanfragen über NFS (für Linux) oder SMB (für Windows) getätigt werden \parencite[Kap. 1, NAS Devices]{gupta.2002}.

Bei einem Daten Request wird die Anfrage an das NAS Gerät weitergeleitet, das die notwendigen Daten an den Server zurück sendet, welcher diese dem Client zur Verfügung stellt.

NAS besitzt einige Vorteile: Es gibt eine deutliche Leistungssteigerung bei den Anwendungsservern, da I/O Operationen komplett auf der dedizierten Hardware ausgeführt werden. NAS Devices besitzen eine wesentlich bessere Skalierbarkeit als \ac{DAS} Geräte, es können so viele neue Netzwerkspeicher zu einem System hinzugefügt werden bis die Netzwerkbandbreite bzw. -anschlüsse ausgelastet sind.

Ebenfalls ist die Fehleranfälligkeit wesentlich geringer, wenn ein Anwendungsserver abstürzt, bleibt der Zugriff auf den NAS im Netzwerk immer noch erhalten.

Management und Einsatz sind auch einfach gehalten, die Geräte konfigurieren sich zu großen Teilen selber und sind mit jedem System ansprechbar.

Trotz all dieser Vorteile gibt es, besonders in großen Netzwerken, auch einige Nachteile. Anfragen von Clients erzeugen eine große Menge Netzwerkverkehr, der viel Bandbreite beansprucht. Sobald viele Request gleichzeitig auftreten, kann es deswegen auch zu Performanceeinbrüchen kommen.

Durch die Zentralisierung des Speichers wird dieser auch anfälliger für bösartige Attacken, Netzwerk Verkehr kann abgefangen oder modifiziert werden \parencite[Kap. 1, Adv. and Disadv. of NAS Devices]{gupta.2002}.


\subsubsection{Storage Area Network (SAN)}
\ac{SAN} wurde entwickelt, um enorme Mengen an Daten zu verarbeiten. Sie werden nicht direkt in das Client- oder Servernetzwerk eingebunden, sondern bilden ein eigenes, das die verschiedenen Speicher integriert. Diese können dann nur durch die SAN Server angesprochen werden. Dies führt zu einem sehr sicheren Setup, da die Geräte vor den Clients komplett verborgen sind \parencite[Kap. 1, SANs]{gupta.2002}.

Im Grunde ist ein \ac{SAN} ein FibreChannel-Netzwerk bei dem Speicher über sogenannte Fabrics, bestehend aus einem oder mehreren Switches, von den Servern angesprochen werden \parencite[S. 11]{tate.2016}.

Es gibt mehrere wichtige Bestandteile: Die oben erwähnten Server stellen den Zugriffspunkt für Applikationsserver im normalen Netzwerk. Innerhalb des Netzwerks kann es verschiedene Speichertypen geben (Disk Storage Systems, Bandlaufwerke, \acs{RAID} oder \acs{JBOD}).
Fibre Channel-Interfaces werden verwendet, um den Speicher mit den Server zu verbinden und sie damit zu externalisieren. Mithilfe von SAN Interconnects können entfernte Speicher ebenfalls eingebunden werden. Switches und Hubs werden benutzt, um verschiedene Speicher zusammenzuschließen.
Außerdem gibt es noch eine ganze Reihe an weiteren Geräten, die für die Kommunikation zwischen verschiedenen Protokollen und Technologien verantwortlich sind \parencite[Kap. 2, SAN Components and Building Blocks]{gupta.2002}.

Ein vereinfachtes Setup kann in \autoref{fig:storageareanetwork} betrachtet werden.

\begin{figure}[hbt]
	\centering
	\includegraphics[scale=0.9]{images/storage-area-network}
	\caption{Typische SAN Konstruktion \parencite[Kap. 1]{gupta.2002}}
	\label{fig:storageareanetwork}
\end{figure}

Durch dieses eigene Netzwerke können einige Vorteile ausgenutzt werden, die so bei \ac{NAS} nicht oder nur schwer zu erreichen sind. Zur Verbindung der Geräte können Hochleistungsnetzwerke wie FibreChannel verwendet werden, die wesentlich performanter sind als Ethernet. Da das SAN getrennt vom Netzwerk der Applikationsserver ist, muss keine Bandbreite geteilt werden, was ebenfalls die Anwendungen schneller macht. Genauso wie bei NAS werden I/O-Aktionen auf die SAN-Geräte verlagert und es können zusätzlich Performance intensive Backups isoliert im SAN erledigt werden \parencite[Kap. 1, SANs]{gupta.2002}.

SANs haben kein eigenes Filesytem, da nur Blockzugriff geboten wird. Entsprechend Fileservern können aber durchaus ein Dateisystem zur Verfügung stellen. SANs bieten einfachen Blockzugriff auf die Daten, für die Server sind sie nur weitere Festplatten. Dadurch ist es einfacher, das SAN beliebig zu skalieren, da einfach nur neue Speichergeräte hinzugefügt werden müssen. Die Ausfallsicherheit wird ebenfalls höher, da auf der einen Seite verschiedene Server Zugriff auf das SAN haben und einfach ein weiterer angesprochen werden kann, falls ein Ausfall stattfindet und auf der anderen Seite eine Mehrfachanbindung des Speichers an den Server existiert, sodass bei Ausfall eine andere Verbindungshardware (Fabric) gewählt werden kann. 

Es gibt optimierte Software zur Verwaltung von SANs, die höhere Performance ermöglichen.

Ein großer Nachteil von SANs ist, dass sie sehr komplex aufzusetzen sind und auch die Kosten wesentlich höher sind als bei NAS Lösungen \parencite[Kap. 2]{gupta.2002}.