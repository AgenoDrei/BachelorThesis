Nach erfolgreicher Einrichtung einer Cloud Node im Spectrum Scale Cluster, kann diese nun mit unterschiedlichen Cloud Speichern verbunden werden.

Bei der Einrichtung wird davon ausgegangen, dass folgende Einstellungen vorgenommen wurden:
\begin{itemize}
	\item Name der NodeClass: \textbf{TCTNodeClass}
	\item Cloud Typ: \textbf{mcstore}
	\item Name des Cloud-Dateisystems: \textbf{fs1}
\end{itemize}

Nach Hochfahren des Clusters kann mit \lstinline|mmhealth node show|, der Status der unterschiedlichen Dienste untersucht werden. Auf einer einzelnen Maschine kann es aber einige Minuten bauen, bis Scale vollständig hochgefahren ist.

Um den Zustand des Cloud Dienstes zu inspizieren, mit ihm zu arbeiten (Datei Sharing / Tiering) oder ihn zu verändern wird das \lstinline|mmcloudgateway| Programm verwendet. Falls dieser nicht automatisch beim Systemstart initialisiert wird, kann er mithilfe von \lstinline|mmcloudgateway service start -N TCTNodeClass| manuell gestartet werden.

\begin{figure}[hbt]
	\centering
	\includegraphics[scale=0.5]{images/scale-status}
	\caption{Beispielhafte Ausgabe von \lstinline|mmhealth| und \lstinline|mmcloudgateway| - mit konfiguriertem Cloudgateway}
	\label{fig:saclestatus}
\end{figure}


\textbf{Einrichtung des Cloud Sharing Accounts}\\
Nun muss zuerst die S3 Verbindung mit \ac{COS} hergestellt werden. Dafür sollte ein Test der Account Daten im Vorhinhein stattfinden, um spätere Probleme bei der Einrichtung verhindern.

Ein IBM Cloud Object Storage Account auf Softlayer kann unter dieser URL \url{https://www.ibm.com/cloud-computing/bluemix/cloud-object-storage} eingerichtet werden. 

Nach erfolgreicher Erstellung kann ein erster Bucket erstellt und Nutzername, Adresse des Endpunkts und Passwort ausgelesen werden. Hierbei gibt es Unterschiede bei der Befehl Syntax zwischen Spectrum Scale und \ac{COS}. 
Der Nutzername wird bei S3 kompatiblen Schnittstellen normalerweise als \textit{AccessKeyId} und das Passwort als \textit{SecretAccessKey} bezeichnet.\\

\begin{lstlisting}[language=bash, caption=Vortest des Cloud Sharing Accounts]
mmcloudgateway account pre-test --cloud-type cleversafe-new --username "<username>" --pwd-file <path/file/your/secretAccessKey> --cloud-url <cos/endpoint>
\end{lstlisting}

Nach Ausführung wird eine Testverbindung aufgebaut, entstehen keine Fehlermeldungen kann nun der Account eingerichtet werden. Die Befehlssyntax benötigt an dieser Stelle aber noch weitere Informationen, die für deinen eigentlichen Test nicht notwendig waren:\\

\begin{lstlisting}[language=bash, caption=Einrichtung des Cloud Sharing Accounts]
mmcloudgateway account create --cloud-nodeclass TCTNodeClass --cloud-name mcstore --cloud-type cleversafe-new --username "<username>" --pwd-file <path/file/your/secretAccessKey> --enable TRUE --cloud-url <cos/endpoint>
\end{lstlisting}

Nach erfolgreicher Ausführung des Befehls ist der Account fertig eingerichtet und man sollte folgende Ausgabe (\autoref{fig:saclestatus}) bei dem Überprüfen des Cloud Services sehen.

\textbf{Export von lokalen Dateien in die Cloud}\\
Um Informationen in den konfigurierten Cloud Account zu exportieren, muss zuerst in das ausgewählte Datei System für die Cloud Knoten gewechselt werden (hier: \lstinline|cd /gpfs/fs1|).

Mit dem bereits bekannten \lstinline|mmcloudgateway| Befehl können nun einzelne oder mehrere Files zwischen \ac{COS} und der lokalen Maschine kopiert werden.

Es besteht die Möglichkeit Metadaten beim Sharing zu erhalten und alle exportierten Files können in einem Manifest festgehalten werden, um später wieder importiert zu werden.

\begin{lstlisting}[language=bash, caption=Export von lokalen Dateien]
mmcloudgateway files export
\end{lstlisting}

\begin{lstlisting}[language=bash, caption=Import von COS Dateien]
mmcloudgateway files import
\end{lstlisting}

Weitere Informationen hierzu befinden sich in \cite[S. 613]{ibmadmin.2017}.

\todo{Aktualiserung nach Auflösung des Export Bugs}