\subsection{Umfeld} \label{subsec:enviroment}

\begin{wrapfigure}{r}{0.35\textwidth}
	\begin{center}
		\includegraphics[width=0.32\textwidth]{images/firma-deckblatt}
	\end{center}
	\caption{Firmenlogo des Ausbildungsbetriebs IBM Deutschland GmbH}
\end{wrapfigure}

Diese Bachelor Arbeit begleitet ein zwölfwöchiges Praktikum bei der Ausbildungsfirma IBM Deutschland GmbH. Bestandteile der Thesis sind die praktischen Implementierungen und die zugehörige Recherche innerhalb des Betriebs. 

Bei IBM wird die Arbeit in dem Bereich EMEA Storage Competence Center (ESCC), speziell in der Abteilung Infrastructure \& Lab Solutions angefertigt. Diese ist im IBM Kelsterbach Standpunkt in Hessen angesiedelt, an welches auch ein Speicherlabor direkt angeschlossen ist.

Aufgabe der Abteilung ist das Management und die Erstellung von Labordemonstrationen mithilfe von verschiedenen Server, Speicher und Softwarelösungen. Diese werden dann verwendet um Funktionalitäten von IBM Speicher Produkten etwaigen Kunden vorzuführen.

\subsection{Ist-Zustand}

In der Abteilungen werden zur Zeit verschiedene lokale Speicherlösungen von IBM in Testumgebungen zur Verfügung gestellt. Auch komplexe Lösungen, wie zum Beispiel Spectrum Scale, können in virtuellen Speicherclustern realistisch simuliert werden. Besonders neue Features sollen hierbei vorgeführt werden.

Aufgrund vieler relativ alter Produkte, die schon seit Jahrzehnten stets weiterentwickelt wurden, gibt es nur wenige mit öffentlichen Cloud Dienstleistern zusammenspielende Demoanwendungen.

Ebenfalls werden vereinzelt ältere Versionen von Software verwendet, die noch keine oder nur teilweise Anbindung an Cloud Dienste ermöglichen. 

Die Anzahl an Demoanwendungen, die direkt in der Cloud laufen und auf die lokalen Lösungen zugreifen, ist leider ebenfalls noch relativ gering. 

\subsection{Soll-Zustand}

Langfristig ist die Bereitstellung viele Demoszenarien für Cloud Szenarien ein wichtiges Ziel der Abteilung. Diese sollen auf der einen Seite die Kompatibilität von IBM Produkten mit verschiedenen Cloud Anbietern verdeutlichen und auf der anderen Seite Synergien mit kognitiven Diensten von IBM aufzeigen.

Das Potential vieler Produkte private, hybride und öffentliche Cloud Computing Lösungen zu ermöglichen soll nicht nur in der Theorie vorhanden sein, sondern hier aktiv präsentiert und implementiert sein.

Dies ist nur möglich, wenn eine größere Anzahl an Demos zur Verfügung gestellt wird, die im besten Fall auf \gls{Bluemix} laufen, damit auch die Fähigkeiten vom IBM Cloud Angebot gezeigt werden können. 

\subsection{Aufgaben}

Im Rahmen dieser Arbeit soll das Clusterspeichersystem Spectrum Scale (früher GPFS) mit einer public Cloud Lösung verknüpft werden. Hierzu sollen die vorhandenen Möglichkeiten der Software für diesen Anwendungsfall untersucht und evaluiert werden.
Die Geeignetste wird innerhalb einer Testumgebung aufgesetzt, konfiguriert und mit einem geeigneten Speicher in der Cloud erweitert werden. 
Sollte sich das Setup als sinnvoll erweisen, besteht die Möglichkeit diese Umgebung in einer Produktivumgebung im Labor einzurichten.

Zusätzlich soll eine Demonanwendung entwickelt werden, die das Potential des oben beschrieben Setups testen kann. Diese Anwendung sollte ebenfalls in \gls{Bluemix} entwickelt werden, sodass eine echte Hybrid Cloud Applikation entsteht. 

Diese soll eine Technik (Cloud Tiering) darstellen, mithilfe der Daten aus einem lokalen System in Cloud betrachtet und bearbeitet werden können.

Die Einbindung von weiteren IBM Diensten ist hierbei erwünscht, damit mögliche Anwendungen für Kunden besser dargestellt werden können.

\subsection{Rahmenbedingungen und Abgrenzung}

Sämtliche Arbeit findet innerhalb der in \autoref{subsec:enviroment} erwähnten Abteilung statt. Hierbei wird ebenfalls Fachwissen von den Mitarbeitern zur Unterstützung und Beratung hinzugezogen.

Software und Hardware zum Aufsetzen von IBM Produkten wird bereitgestellt, hierbei speziell IBM Spectrum Scale. Für die Entwicklung mithilfe einer Testumgebung wird ein Lenovo Thinkpad T440 mit Redhat 7.3 Linux verwendet. Hierbei wird der Clusterspeicher in einer mit KVM betriebenen virtuellen Maschine aufgesetzt.

Zur Versionsverwaltung der Demoanwendung wird \gls{Git} verwendet. Notwendiges Projektmanagement wird agil umgesetzt und es werde, wenn notwendig Stories oder auch Tasks erzeugt, die dann zusammen mit den Git Issues getrackt werden können (ein Feature von der Gitumgebung von Bluemix).

Zudem werden eine Reihe externer Tools und Frameworks eingesetzt, ebenso wie IBM Software, die verwendet, angepasst und nicht direkter Teil dieser Arbeit sind. Ich beschränke den Bericht auf von mir gemachte Entwicklung, außer diese steht im direkt Zusammenhang mit Features, die von Anderen implementiert wurden und aus Verständlichkeitsgründen nicht ausgelassen werden können.

IBM Spectrum Scale wird nicht komplett aufgesetzt, da dies zeitaufwendig und bereits häufig gemacht worden ist. Stattdessen wird ein konfiguriertes Abbild verwendet, dass nur noch für die Verwendung mit der Cloud verändert werden muss.

Ich erhebe keinen Anspruch auf die Arbeit Anderer, auch wenn an manchen Stellen die Vorarbeit von Kollegen aus der Abteilung verwendet wird. Entsprechende Stellen werden eindeutig gekennzeichnet und angemessen begründet.