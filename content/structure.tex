Der Aufbau der Arbeit sollte anhand der Gliederung leicht erkennbar sein, wird hier aber noch einmal im Detail dargestellt. Im \autoref{ch:introduction} wird die Motivation der Arbeit, ihr Kontext und die spezifische Aufgabenstellung dargelegt.

Im \autoref{ch:background} werden verschiedene heutzutage relevante Technologie für Speicher, verteilte Speichersysteme und Cloud Computing diskutiert. Sie bilden die Grundlage für die Auswahl der relevanten Tools, speziell im Cloud-Umfeld, für die Umsetzung der Aufgaben unter Berücksichtigung der besonderen Vorzüge des Ausbildungsunternehmens.
An dieser Stelle werden auch sämtliche relevanten Begriffe, Konzepte und Lösungen gezeigt, die notwendig sind, um sich in das behandelte Themenumfeld einzuarbeiten.

Eingehend auf oben erwähnte Beschränkung wird dann im \autoref{ch:methods} eine Auswahl an Umgebungen und Werkzeugen getroffen und verwendete Techniken zum Entwickeln beschrieben.

Resultate und der Ablauf der Entwicklung werden im \autoref{ch:realization} beschrieben. Dabei werden sowohl überwundene Probleme, wie auch die Gesamtarchitektur der Lösung dargelegt.

Zum Abschluss der Arbeit (\autoref{ch:conclusion}) werden diese Ergebnisse kritisch diskutiert und es wird ein Ausblick auf mögliche zukünftige Entwicklungen im Bereich des Hybrid Cloud Computing gegeben.