\chapter{Stand der Technik}\label{ch:background}

Zum Aufsetzen einer Hybrid Cloud Umgebung und dem Entwickeln einer auf dieser laufenden Anwendungen werden verschiedene Grundlagen benötigt. Grundwissen über die unterschiedlichen Speichertypen ist notwendig, um die Herausforderungen eines verteilten Speichersystems zu verstehen. Ebenfalls ist fundiertes Wissen über die Cloud, ihre Umsetzung und Typen wichtig, um in der Lage sein, effiziente Anwendungen für diese zu entwickeln.

In diesem Kapitel werden diese Hintergründe erläutert.

\section{Speicher Lösungen}
Speicherung von Daten ist ein wichtiges Thema in der IT, wodurch unzählige Technologien entstanden sind. Hier werden erst Hardware Speicher beschrieben und danach verschiedene Arten, diesen innerhalb von Rechnernetzwerken zusammenzustellen.

\begin{figure}[hbt]
	\centering
	\includegraphics[scale=0.75]{images/storage-pyramide}
	\caption{Speicher Pyramide \parencite{kaufmann.2016}}
	\label{fig:storagepyramide}
\end{figure}

Prinzipiell kann Speicher immer in eine Art Hierarchie eingeteilt werden. Extrem häufige verwendete Daten landen auf der Ersten Ebene. Diese wird meistens in Form von Cache oder RAM bereitgestellt und bietet extrem kurze Latenz, aber nur geringen Speicherumfang.

Auf der nächsten Ebene stehen schnelle Speichermedien, zum Beispiel SSDs. Direkt darunter ordnen sich klassische Festplatten Speicher ein, die den Großteil des Speichervolumens zur Verfügung stellen. Sie haben höhere Latenzen sind aber dafür günstig und haben eine hohe Kapazität \parencite[Kap. 2, What is Computer Storage?]{kaufmann.2016}.

Mit der letzten Stufe ``Cloud Speicher'' wird sich intensiver in der \autoref{sec:cloud} auseinander gesetzt. \todo{Check Reference}

\subsection{Hardware}

Unabhängig von der Zusammenstellung umfänglicher Speicherlösungen, müssen die Daten trotzdem irgendwann auf einen Hardwarespeicher geschrieben werden. Folgender Abschnitt erläutert, die allgemeine Funktion und geht auf einige der häufigsten verwendeten Typen ein.

\subsubsection{Hard Disk Drive (HDD)}

Hierbei handelt es sich um einen nicht volatilen Speicher, der digital kodierten Inhalte auf der magnetischen Oberfläche einer Platte speichert. Traditionell gibt es zwei Protokolle: SAS im kommerziellen und SATA im privaten Sektor \parencite{wikibooks.2016}.

Ein oder mehrere Leseköpfe werden mithilfe von Armen und einem Servomotor über die Platte bewegt, um relevante Stellen auszulesen. Bei der Herstellung von HDDs werden Tracks in die Platte geschrieben, die jeweils ein Label und Blockinformationen besitzen. Durch sie ist es einfacher, den Lesekopf zu platzieren. Jeder dieser Tracks kann bis zu einigen MBs an Daten beinhalten. Durch die Blockbildung bei I/O-Operationen kann häufig nur ein einzelner Block (64-256) geschrieben werden, wodurch Festplatten unnötig langsam werden. Durch neue Technologie können IOs in eine Warteschlange eingereiht werden, sodass die Platte kontinuierlich lesen und schreiben kann. Trotzdem ist nur eine maximale Geschwindigkeit von ungefähr 300 \gls{IOPS} das Limit für Platten mit 10.000 RPM \parencite[Kap. 3]{kaufmann.2016}.

Jede Festplatte hat heutzutage einige private Sektoren, auf denen Reparaturinformationen und Reserveblöcke gespeichert sind. Fällt ein Block in der Festplatte aus (durch zum Beispiel physikalische Beschädigung), kann dieser fließend ersetzt werden, sodass keine Daten verloren gehen. Kurz vor dem Ausfall einer Platte steigt diese Anzahl der Lesefehler einer Platte massiv an, sodass eine Datenreovery versucht werden kann. Leider ist dies keine verlässliche Methode, um Daten zu sichern, sodass andere Methoden verwendet werden sollten \parencite[Kap. 3]{kaufmann.2016}.

Um Datenverlust durch Plattenausfälle entgegenzuwirken, besteht die Möglichkeit der Verwendung von so genannten RAIDs. Dies ist eine Methode, bei der mehrere HDDs so zusammengeschaltet werden, sodass bei dem Ausfall einer Einzelnen die Daten weiterhin verfügbar sind \parencite{wikibooks.2016}.

Festplatten sind gut erforscht, es treten extrem selten unbekannte Probleme auf und werden seit Jahrzehnten verwendet, ihr Speicher ist extrem günstig geworden und in Masse verfügbar. Ihr größter Nachteil ist eine sehr geringe I/O-Geschwindigkeit im Vergleich mit neueren Technologien \parencite[Kap. 3]{kaufmann.2016}.

\subsubsection{Solid State Drive (SSD)}

Im Jahr 2007 wurde der erste SSD-Speicher vermarktet und stach durch extrem hohe Zugriffszahlen heraus. Diese waren bei 4000 \gls{IOPS} bereits um das zehnfache höher als die klassischer Festplatten. Heutzutage ist es möglich IOPS Zahlen im Millionenbereich zu erreichen, was eine Revolution für Speicher bedeutet \parencite[Kap. 3]{kaufmann.2016}.

\begin{figure}[hbt]
	\centering
	\includegraphics[scale=0.85]{images/flash}
	\caption{Funktionsweise eines NAND-Flashspeichers \parencite{kaufmann.2016}}
	\label{fig:flash}
\end{figure}

SSDs benutzen Nicht-Und (NAND) memory chips zum Speichern einzelner Bitinformationen. Von diesen Zellen gibt es dann einige Millionen, die beliebig adressiert werden können (Random Access Memory). Innerhalb eines so genannten Floating Gate können nun Elektronen eingefangen werden, um eine binäre Eins zu repräsentieren. Im Ausgangszustand sind diese ``Fallen'' leer, um einen Nullinformationen zu speichern. Sobald der Speichercontroller eine Zelle laden will, wird der Einlasstransistor geöffnet und einige Elektronen in der Zelle für eine beliebig lange Zeit gefangen.

Dieser Prozess findet in \autoref{fig:flash} statt, wenn auf der Bit- und Wordline eine positive Ladung gesetzt wird. Meistens liegen mehrere Zellen auf einer Reihe, was den Leseprozess verkomplizieren kann. Es wird eine niedrige Spannung an alle Wordlines angelegt, wodurch bei geladenen Zellen eine Konduktivität festgestellt werden kann. 

Wird eine einzelne Zelle zu häufig beschrieben, kann diese Isolation verlieren, wodurch sie unnutzbar zum Speichern wird. Schreibstrategien und Reservezellen werden verwendet, um dem vorzubeugen.	

Zum Löschen der Zellen wird eine hohe negative Spannung an die Zellen angelegt, wodurch die gefangenen Elektronen aus den Floating Gates heraus gestoßen werden. Dies passiert meistens in fest definierten Pages, sodass keine speziellen Blöcke gelöscht werden können. Nach dem Löschen eines Blockes wird dessen Zeiger auf eine vorher geleerte Page bewegt und der Block als ``discared'' markiert. Sind innerhalb einer Page genug Blöcke in diesem Zustand, werden die verbleibenden Daten verschoben und die gesamte Page gelöscht und Teil des freien Speichers der SSD \parencite{kaufmann.2016}. 
Durch diesen Prozess ist das Löschen von Dateien nicht mehr zuverlässig, da die Daten noch für lange Zeit innerhalb eines ``discared'' Blockes liegen können, der erst wesentlich später vom Controller gelöscht wird.

SSDs können über das SATA-Protokoll angeschlossen werden, aber in den letzten Jahren wurde das wesentlich performantere NVMe entwickelt.

Da dieser Speicher keine sich bewegenden Teile hat, ist er wesentlich robuster was Schläge, Vibrationen und hohe Temperaturen angeht. Ebenfalls ist der Strombedarf wesentlich geringer, was zusammen mit den vorherigen Punkten den Betrieb von SSDs in großen Rechenzentren wesentlich einfacher und günstiger macht.

Trotzdem gibt es auch einige Nachteile von Flashspeicheren: Die meisten Anwendungen sind nicht darauf ausgelegt, so hohe Zugriffszahlen zu unterstützen. In der Vergangenheit hat I/O immer ein Bottleneck in Applikationen dargestellt, sodass die meisten Programme entsprechend designed wurden.
Dies zieht sich durch die gesamte Softwarelandschaft und auch die geläufigen Betriebssysteme sind nicht in der Lage, diese neue Geschwindigkeit voll auszunutzen. Ebenfalls ist das SATA Interface nicht perfekt für SSDs, da es ursprünglich dafür designed war, die niedrigen Bandbreiten von Festplatten auszugleichen \parencite[Kap. 3]{kaufmann.2016}.


\subsubsection{Bandlaufwerk}

Bandlaufwerke funktionieren im Grunde wie die früher verwendeten Kassetten. Innerhalb des Laufwerkes befindet sich ein dünner Streifen aus Plastik mit einer magnetischen Oberfläche. Dieser wird, ähnlich wie bei Festplatten, mit digitalen Daten beschrieben. 

Es kann nicht auf beliebige Bereiche des Bandes zugegriffen werden, Lesen und Schreiben erfolgt immer sequentiell. Das Laufwerk muss von Anfang bis Ende durchlaufen werden, um I/O Aktionen durchführen zu können \parencite{adrc.2009}.

Ein Bandlaufwerk verwendet einen kleinen Motor, um das Band auf- und abzuwickeln. Dabei wandert dieses entlang eines Lese- und Schreibkopfes. Um Unterschiede zwischen der Geschwindigkeit der ankommenden Daten vom Computer und der begrenzten \gls{IOPS} des Laufwerkes auszugleichen, wird eine Steuereinheit verwendet. Diese regelt Fehlerhandling, Puffer und andere logische Operationen.

Informationen werden in die Steuereinheit geladen und dann auf das Band geschrieben. Dieser Prozess wiederholt sich solange, bis keine Daten mehr vorhanden sind.

Aufgrund von geringen Kosten und hoher Lebenszeit wird diese Art von Laufwerk immer noch häufig verwendet. Besonders of wird es für Backupszenarien genutzt, da hier die Nachteile des nur sequentiellen Zugriffes auf Daten keine große Rolle spielen \parencite{adrc.2009}. 

\subsection{Speicheranbindung über Netzwerke}
\subsubsection{Direct Attached Storage (DAS)}

Bei \ac{DAS} handelt es sich um externen Speicher, der direkt mit einem oder mehreren Servern über ein \gls{SCSI} Interface verbunden ist. Dabei wird kein Netzwerk verwendet. Es gibt verschiedene Typen von \ac{DAS}, bei manchen werden nur beliebige Platten zusammengewürfelt (\ac{JBOD}), bei anderen ganze Plattenarrays angeschlossen.
Hierbei handelt sich um den ältesten Speichertypen, der lange vor dem Entstehen von SAN oder NAS verwendet wurde.

Es kann nur eine begrenzte Anzahl von \ac{DAS} an einen Rechner angeschlossen werden, da die Anzahl von SCSI-Schnittstellen an Servern limitiert ist. Dadurch ist diese Art von Speicher leider nur begrenzt skalierbar.

Ein weiterer großer Nachteil von direkt angebundenen Speicher ist, dass bei einem Ausfall des Servers ein wesentlicher Teil der Daten nicht mehr erreichbar ist. Die Verfügbarkeit der Daten hängt also nicht nur vom Speicher selber ab, sondern auch von dem bereitstellenden Server.

Wird das Laufwerk an mehrere Server angeschlossen, um obige, Problem entgegenzuwirken, erhöht sich durch einen Ausfall die Zugriffszeit beträchtlich \parencite[Kap. 1, Disk Storage Systems]{gupta.2002}.

Der größte Vorteil von DAS Geräten ist, dass sie einfach aufzusetzen und zu warten sind, da keine Netzwerkkenntnisse oder Hardware benötigt werden. Dies bedeutet geringe Kosten und einfache Handhabung \parencite{beal.2017}.

\subsubsection{Network Attached Storage (NAS)}

\ac{NAS} ist ein einfaches System, um Daten an einer einheitlichen Stelle im Netzwerk zu speichern. Hierdurch werden die Daten von den Servern selber zu speziell zur Speicherung ausgelegten Hardware wegbewegt. 

Ein NAS-Gerät ist spezialisierte Hardware mit meist zugehöriger Management Software. In den meisten Fällen muss es nur mit dem Firmennetzwerk verbunden und eingeschaltet werden. Ebenfalls ist der Speicher plattformunabhängig und kann mithilfe verschiedener Protokolle von jedem Client verwendet werden. Einige Beispiele für diese sind: HTTP (Internetanfragen), FTP oder TCP/IP. Ebenfalls können Datenanfragen über NFS (für Linux) oder SMB (für Windows) getätigt werden \parencite[Kap. 1, NAS Devices]{gupta.2002}.

Bei einem Daten Request wird die Anfrage an das NAS Gerät weitergeleitet, das die notwendige Daten an den Server zurück sendet, welcher diese dem Client zur Verfügung stellt.

NAS besitzt einige Vorteile: Es gibt eine deutliche Leistungssteigerung bei den Anwendungsservern, da I/O Operationen komplett auf der dedizierten Hardware ausgeführt werden. NAS Devices besitzen eine wesentlich bessere Skalierbarkeit als \ac{DAS} Geräte, es können beliebig viele neue Netzwerkspeicher zu einem System hinzugefügt werden.

Ebenfalls ist die Fehleranfälligkeit wesentlich geringer, wenn ein Anwendungsserver abstürzt, bleibt der Zugriff auf den NAS im Netzwerk immer noch erhalten.

Management und Einsatz sind auch einfach gehalten, die Geräte konfigurieren sich zu großen Teilen selber und sind mit jedem System ansprechbar.

Trotz all dieser Vorteile gibt es, besonders in großen Netzwerken, auch einige Nachteile. Anfragen von Clients erzeugen eine große Menge Netzwerkverkehr, der viel Bandbreite beansprucht. Sobald viele Request gleichzeitig auftreten, kann es deswegen auch zu Performanceeinbrüchen kommen.

Durch die Zentralisierung des Speichers wird dieser auch einfacher anfälliger für bösartige Attacken, Netzwerk Verkehr kann abgefangen oder modifiziert werden \parencite[Kap. 1, Adv. and Disadv. of NAS Devices]{gupta.2002}.


\subsubsection{Storage Area Network (SAN)}
\ac{SAN} wurden entwickelt, um enorme Mengen an Daten zu verarbeiten. Sie werden nicht direkt in das Client- oder Servernetzwerk eingebunden, sondern bilden ein eigenes, das die verschiedenen Speicher miteinander verbindet. Dieses kann dann nur durch die SAN Server angesprochen werden. Dies führt zu einem sehr sicheren Setup, da die Geräte vor den Clients komplett verborgen sind \parencite[Kap. 1, SANs]{gupta.2002}.

Es gibt mehrere wichtige Bestandteile: Die oben erwähnten Server stellen den Zugriffspunkt für Applikationsserver im normalen Netzwerk. Innerhalb des Netzwerks kann es verschiedene Speichertypen geben (RAIDs, JOBDs, Disk Storage Systems, Bandlaufwerke).
Interfaces werden verwendet, um den Speicher mit den Server zu verbinden und sie damit zu externalisieren. Mithilfe von SAN Interconnects können entfernte Speicher ebenfalls eingebunden werden. Hubs und Router werden benutzt, um verschiedene Speicher zusammenzuschließen, wobei Router das Signal immer nur zu einem Gerät weiterleiten und es nicht broadcasten.
Außerdem gibt es noch eine ganze Reihe an weiteren Geräten, die für die Kommunikation zwischen verschiedenen Protokollen und Technologien verantwortlich sind \parencite[Kap. 2, SAN Components and Building Blocks]{gupta.2002}.

Ein vereinfachtes Setup kann in \autoref{fig:storageareanetwork} betrachtet werden.

\begin{figure}[hbt]
	\centering
	\includegraphics[scale=0.9]{images/storage-area-network}
	\caption{Typische SAN Konstruktion \parencite[Kap. 1]{gupta.2002}}
	\label{fig:storageareanetwork}
\end{figure}

Durch dieses eigene Netzwerke können einige Vorteile ausgenutzt werden, die so bei \ac{NAS} nicht oder nur schwer zu erreichen sind. Zur Verbindung der Geräte können Hochleistungsnetzwerke wie FibreChannel verwendet werden, die wesentlich performanter sind als Ethernet. Da das SAN getrennt vom Netzwerk der Applikationsserver ist, muss keine Bandbreite geteilt werden, was ebenfalls die Anwendungen schneller macht. Genauso wie bei NAS werden I/O-Aktionen auf die SAN-Geräte verlagert und es können zusätzlich Performance intensive Backups isoliert im SAN erledigt werden \parencite[Kap. 1, SANs]{gupta.2002}.

Das Filesystem von SANs wird nicht von den Geräten selber verwaltet, sondern von den entsprechenden Fileservern. SANs bieten einfachen Blockzugriff auf die Daten, für die Server sind sie nur weitere Festplatten. Dadurch ist es einfacher, das SAN beliebig zu skalieren, da einfach nur neue Speichergeräte hinzugefügt werden müssen. Die Ausfallsicherheit wird ebenfalls höher, da meistens verschiedene Server Zugriff auf das SAN haben und einfach ein weiterer angesprochen werden kann, falls ein Ausfall stattfindet.

Es gibt optimierte Software zur Verwaltung von SANs, die höhere Performance ermöglichen.

Ein großer Nachteil von SANs ist, dass sie sehr komplex aufzusetzen sind und auch die Kosten wesentlich höher sind als bei NAS Lösungen \parencite[Kap. 2]{gupta.2002}.

\newpage

\section{Dateisysteme}

Für Anwendungen und Nutzer organisiert ein Dateisystem eine abstrahierte Sicht auf sämtlichen Daten. Es stellt eine Schnittstelle zur Verfügung für Online-, Offline- und Programmzugriff. Ein Filesystem besteht aus einer Sammlung aller Dateien und einer Ordnerstruktur, die erstere organisiert und beschreibt. 

Das Betriebssystem ordnet Speicherblöcke logischen Einheiten zu, so genannten \textbf{Files}. Die physikalischen Eigenschaften des Speichergerätes werden hiermit vor dem Nutzer verborgen. SSDs, HDDs, Online Speicher und sogar Cluster-Dateisysteme können hierdurch auf die gleiche Art betrachtet werden \parencite{silberschatz.2012}.

Im Folgenden werden verschieden Clustersysteme, ihre Vor- und Nachteile untersucht. Dabei wird besonderer Wert auf die Kompatibilität mit Cloud Lösungen gelegt. Ebenfalls liegt ein stärker Fokus auf der Lösung von IBM, die Gründe hierfür werden im \autoref{ch:methods} näher erläutert.

\subsection{Speicher Netzwerke}
\subsubsection{SAN}
\subsubsection{NAS}
\subsubsection{NFS}

\subsection{Parallele Speichersysteme}
\subsubsection{IBM Spectrum Scale - GPFS}
\subsubsection{Hadoops HDFS}
\subsubsection{Google File System - GFS}

\section{Cloud Computing}\label{sec:cloud}
%Insert General explanation here
\todo{Weitere Quellen benutzen}

Cloud Computing kann als eine neue Art der Bereitstellung von Programmen, Diensten und Infrastruktur betrachtet werden, die häufig virtualisiert und automatisch skalierbar ist. Es ist die logische Weiterentwicklung von Netzwerk oder Internet Computing und hilft dabei vorhandene Ressourcen besser auszunutzen. 
Software, Laufzeiten und sogar Plattformen werden in einem verteilten Serverclusters angeboten. Diese können dann von jeder Art  Client Geräten (Personal Computer, Smartphones, \ac{IoT} Devices) angesprochen werden. Außer Anwendungen werden häufig  auch Speicher, bestimmte Dienste oder Entwicklungswerkzeuge über die Cloud angeboten \parencite[S. 3]{furth.2010}.

Ein großer Vorteil von Cloud Umgebung ist die Dezentralisierung von Anwendungen. Statt für eine Applikation einen einzelnen Server zu nutzen, wird diese in einem Netz von diesen virtualisiert. Dieses Netz von Servern kann eine Vielzahl von Anwendungen provisonieren und bei Bedarf automatisch höhere Leistung zuordnen. Hierdurch werden Kosten gespart und eine bessere Verfügbarkeit garantiert \parencite[S. 7]{furth.2010}.

Es gibt einige Schlüsseltechnologien die Cloud Computing erst ermöglichen. 

Zum einen ist Virtualisierung extrem wichtig, da Container und virtuelle Maschinen auf verschiedenen Servern gehostet werden können (Entkopplung von Anwendung, OS, Hardware und Nutzerdaten), um unterschiedliche Anwendungen für globale Zugriffspunkte bereitzustellen. 
Sobald mehr Performance notwendig ist (mehr Zugriffe) können diesen mehr Serverleistung bekommen oder es werden weitere Instanzen auf zusätzlichen Rechnern erzeugt.
Um diese effektiv umzusetzen besitzen Cloud Systeme einen Load Balancer, der automatisch Zugriffe auf die bereitstellenden Server verteilt und dafür sorgt, dass zum Beispiel Sitzungen aufrecht erhalten werden (Die Zugriffe eines Nutzer landen immer auf dem selben Server).
Zur Loslösung von Speicher wird dieser gerne in separaten \acs{SAN} ausgelagert \parencite[S. 22]{rafaels.2015}.
\subsection{Cloud Modelle}
%On/Off Premise
%SaaS / PaaS / IaaS

Es gibt verschiedene Cloud Modelle, die jeweils unterschiedliche Features für den User bereitstellen. Hauptsächlich unterscheidet man hier zwischen \ac{IaaS}, \ac{PaaS} und \ac{SaaS}, welche aber auch gleichzeitig vom selben Anbieter zur Verfügung stehen können.

\begin{itemize}
	\item \textbf{IaaS}:\\
	Dem Nutzer hat die Kontrolle über das zu verwendende OS, den Speicher und gehostete Anwendungen. Nur die unterliegende Cloud Infrastruktur ist nicht veränderbar und häufig sind auch nur bestimmte Teile der Firewall anpassbar.
	\item \textbf{PaaS}:\\
	In diesem Fall hat der Nutzer keine Wahl was Betriebssystem, Netzwerk Fähigkeiten oder Speicher angeht, es existiert eine vorinstallierte Umgebung (Datenbanken, Frameworks, Laufzeiten, usw.) zum Ausführen von eigenen Applikationen. Begrenzt bestehen Konfigurationsmöglichkeiten für die Hosting Umgebung der Software. 
	\item \textbf{SaaS}:\\
	Es wird nur eine bestimmte Anwendung oder Interface dem Nutzer offengelegt. Er kann diese verwenden und ggf. seine persönliche Instanz konfigurieren, hat aber keinerlei Kontrolle über die unterliegend Laufzeit Umgebungen, den Speicher, das Netzwerk oder das OS \parencite[S. 15f]{rafaels.2015}.
\end{itemize}

Des weiteren können Cloud Server in verschiedenen Umgebungen installiert werden, um Zugriff, Sicherheit und Verfügbarkeit zu regulieren.

\begin{itemize}
	\item \textbf{Public Cloud}:\\
	Die Cloud Server sind öffentlich zugänglich und jeder kann Anwendungen erstellen und über die Infrastruktur bereitstellen. Das System wird meistens von einer Firma oder Regierungsstelle betrieben und die notwendige Hardware steht auf dessen Gelände. 
	\item \textbf{Private Cloud}:\\
	Die gesamte Cloud Infrastruktur wird nur für einen Kunden zur Verfügung gestellt und ist nur für diesen zugänglich. Sie kann von ihm oder einer dritten Gruppen gemanagt werden. Es besteht die Möglichkeit das System auf dem Gelände des Käufers zu installieren, um zum Beispiel Datensicherheit von kritischen Informationen zu gewährleisten. 
	\item \textbf{Hybrid Cloud}:\\
	Eine Hybrid Cloud verbindet Elemente aus einem öffentlichen und privaten Setup. Zum Beispiel können Teile der Berechnung in der public Cloud stattfinden und Personaldaten auf den privaten Servern verbleiben. Hierdurch können Datenschutz Bedenken berücksichtigt und die Vorteile öffentlichen Cloud Computings ausgenutzt werden. Zur Kommunikation zwischen den beiden Server Clustern können verschiedene standardisierte Schnittstellen verwendet werden (z.B. REST)  \parencite{rafaels.2015}.
\end{itemize}

\subsection{Anbieter}
Durch die Popularität von Cloud Computing gibt es mittlerweile an Vielzahl von Anbietern, an dieser Stelle werden einige der wichtigsten kurz vorgestellt.

\textbf{IBM Bluemix}\\
\textbf{Microsoft Azure}\\
\textbf{Amazon Web Services}\\

\subsection{Entwicklung in einem Cloud System}