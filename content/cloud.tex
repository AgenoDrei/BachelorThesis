\subsection{Cloudmodelle}
\todo{Weitere Quellen benutzen}

Es gibt verschiedene Cloudmodelle, die jeweils unterschiedliche Features für den User bereitstellen. Hauptsächlich unterscheidet man hier zwischen \ac{IaaS}, \ac{PaaS} und \ac{SaaS}, welche aber auch gleichzeitig vom selben Anbieter zur Verfügung gestellt sein können.

\begin{itemize}
	\item \textbf{IaaS}:\\
	Der Nutzer hat die Kontrolle über das zu verwendende OS, den Speicher und gehostete Anwendungen. Nur die unterliegende Cloudinfrastruktur ist nicht veränderbar und häufig sind auch nur bestimmte Teile der Firewall anpassbar.
	\item \textbf{PaaS}:\\
	In diesem Fall hat der Nutzer keine Wahl was Betriebssystem, Netzwerkfähigkeiten oder Speicher angeht, es existiert eine vorinstallierte Umgebung (Datenbanken, Frameworks, Laufzeiten, usw.) zum Ausführen von eigenen Applikationen. Begrenzt bestehen Konfigurationsmöglichkeiten für die Hosting Umgebung der Software. 
	\item \textbf{SaaS}:\\
	Es wird nur dem Nutzer eine bestimmte Anwendung oder Interface offengelegt. Er kann diese verwenden und ggf. seine persönliche Instanz konfigurieren, hat aber keinerlei Kontrolle über die unterliegende Laufzeitumgebungen, den Speicher, das Netzwerk oder das \acs{OS} \parencite[S. 15f]{rafaels.2015}.
\end{itemize}

Außerdem können Cloudsysteme in verschiedenen Umgebungen installiert werden, um Zugriff, Sicherheit und Verfügbarkeit zu regulieren.

\begin{itemize}
	\item \textbf{Public Cloud}:\\
	Die Cloudserver sind öffentlich zugänglich und jeder kann Anwendungen erstellen und über die Infrastruktur bereitstellen. Das System wird meistens von einer Firma oder Regierungsstelle betrieben und die notwendige Hardware steht auf dessen Gelände. 
	\item \textbf{Private Cloud}:\\
	Die gesamte Cloudinfrastruktur wird nur für einen Kunden zur Verfügung gestellt und ist nur für diesen zugänglich. Sie kann von ihm oder einer dritten Gruppe gesteuert werden. Es besteht die Möglichkeit, das System auf dem Gelände des Käufers zu installieren, um zum Beispiel Datensicherheit von kritischen Informationen zu gewährleisten. 
	\item \textbf{Hybrid Cloud}:\\
	Eine Hybrid Cloud verbindet Elemente aus einem öffentlichen und privaten Setup. Zum Beispiel können Teile der Berechnung in der Public Cloud stattfinden und Personaldaten auf den privaten Servern verbleiben. Hierdurch können Datenschutzbedenken berücksichtigt und die Vorteile öffentlichen Cloud Computings ausgenutzt werden. Zur Kommunikation zwischen den beiden Serverclustern können verschiedene standardisierte Schnittstellen verwendet werden (z.B. REST)  \parencite{rafaels.2015}.
\end{itemize}

\subsection{Cloudanbieter} \label{subsec:cloudprovider}
Durch die Popularität von Cloud Computing gibt es mittlerweile eine Vielzahl von Anbietern, an dieser Stelle werden nur die drei Größten (1. Amazon Web Services, 2. Microsoft, 3. IBM) kurz vorgestellt \parencite{statistia.2016}. Es wird sich hierbei auf Anbieter beschränkt, die mindestens \acs{PaaS} Funktionalitäten bieten, da für das Projekt dieser Arbeit eigene Anwendungen entwickelt werden müssen.


\textbf{Amazon Web Services}\\
Amazon ist im Moment der Marktführer mit fast 30 Prozent Marktanteil. Dies basiert zum einen darauf, dass \ac{AWS} mit einer der ersten modernen Anbieter (seit 2002 in einfacher Form und seit 2006 mit den ersten Cloud Services) ist und es eine Vielzahl von Diensten gibt. Es gibt sowohl direkt ansprechbare Services für Speicher, Rechenleistung und Netzwerkfähigkeiten für \acs{PaaS} Aufgaben als auch VM Systeme und Container für frei konfigurierbare Anwendungen (\acs{IaaS}).

Zum Beispiel die \ac{S3}-Schnittstelle wurde von Amazon entwickelt und wird an einer Vielzahl von Stellen für die Kommunikation mit Objektspeichersystemen verwendet \parencite{aws.2017}.

Aufgrund Amazons langer Geschichte mit Cloud Computing bieten sie im Moment die meisten Datencenter weltweit an und haben mehr Rechenleistung als die beiden anderen Konkurrenten zusammen. Auch bei der Auswahl von Diensten haben Entwickler hier die meiste Vielfalt. Als Nachteil gelten eine gewisse Komplexität bei der Nutzung und eine Vernachlässigung des Hybrid- und Privat Cloud-Angebots \parencite{computerworlduk.2016}.


\textbf{Microsoft Azure}\\
Azure ist die Anwendungsplattform von Microsoft für eine öffentliche Cloud mit \acs{PaaS}- und \acs{IaaS}-Features. Sie ist spezialisiert für die Kombination mit Microsoft Diensten, Programmiersprachen und Lösungen. Mithilfe eines Management Portals lassen sich gehostete Applikationen verwalten. 

Es besteht die Möglichkeit, \acs{VM}s oder Web Applikationen, für die eine Bandbreite an verschiedenen Diensten zur Verfügung steht, zu erzeugen. Es können ganze \acs{VM}s, Container oder auch nur einzelne lokale Komponenten (z.B. Speicher )aus on-premise Datencentern leicht in die Cloud gehoben werden.

Insgesamt gibt es viele Überschneidungen mit \acs{AWS}, aber die Kompatibilität mit Microsoft/Windows-Komponenten ist wesentlich höher und andere Lösungen sind weniger vertreten \parencite{microsoft.2015}.


\textbf{IBM Bluemix}\\
Bei Bluemix handelt es sich um die \acs{PaaS}/\acs{IaaS}-Lösung von IBM (wobei diese Funktion von Softlayer zur Verfügung gestellt werden), die zuerst 2014 das erste Mal vorgestellt wurde. Außer der Möglichkeit, \acs{VM}s, Root Server und Docker Container zu hosten, werden eine Vielzahl von IBM und Drittpartei-Dienste angeboten, die von Speicher bis zu \ac{IoT} oder Maschinenlernen reichen.	

Zur Bereitstellung von Applikationen wird Cloud Foundry verwendet, eine Open Source-Anwendungsplattform zur Abstraktion von Infrastruktur \parencite{bluemix.2017}.

Es kann in einer Vielzahl populärer Sprachen entwickelt werden (Node, PHP, Ruby, Java, Go...) und es gibt sogenannte Buildpacks, die nicht vorhandene Laufzeitumgebungen zum Teil nachrüsten. Ebenfalls kann Bluemix als private Cloud oder in einem "dedizierten" Modus zur Verfügung gestellt werden. Bei diesem wird garantiert, dass eine \acs{VM} nur auf einem isolierten Bare-Metal Server läuft.

In der Bluemix-Konsole (Managementoberfläche) lassen sich auch einige \gls{DevOps} Funktionen konfigurieren. Es gibt einen eigenen Code Editor, Gitlab Integration und eine Buildpipeline für \ac{CI} \parencite{fassnacht.2016}.

Im Vergleich zur Konkurrenz bietet IBM viele \acs{IaaS} Dienste, kann aber bei Funktionen für Entwickler von Anwendungen nicht mithalten \parencite{computerwoche.2016}. 

%\subsection{Entwicklung in einem Cloud System}