\chapter{Methoden}\label{ch:methods}

In diesem Kapitel werden notwendige Komponenten der zu entwickelnden Hybrid Cloud Lösungen ausgewählt und begründet. Ebenfalls werden Techniken beschrieben, die bei der Entwicklung der Demoapplikation verwendet werden. 

Einschränkungen bei der Wahl von Lösungen sind aufgrund der Umgebung \ref{subsec:enviroment} vorhanden und werden entsprechend hier gekennzeichnet und diskutiert.

\section{Auswahl des verteilten Dateisystems}
Bei der Wahl des Dateisystems müssen die ursprüngliche Anwendungsszenarien von Hadoop und Spectrum Scale betrachtet werden. 

Apache Hadoop soll dafür verwendet werden Operationen auf möglichsten effiziente Weise in einem verteilten System ausführen. \acs{HDFS} stellt hierbei die Speicherkomponente dar und baut auf normalen ext4 Dateisystemen auf. Dies macht insbesondere Sinn, da Hadoop Cluster meistens aus gewöhnlicher Hardware bestehen. Nachteil hiervon ist ein erschwerter Zugriff, da auf das System mit einem speziellen Set Befehlen immer von außen zugegriffen und eine Datenveränderung erst innerhalb der geschachtelten Systeme ausgetauscht werden muss.
Hadoop ist Open Source und kann somit kostenlos verwendet werden %\parencite{snowflake.2016}. 

Spectrum Scale hingegen war ursprünglich als proprietärer High-Performance Speicher System für jede Art von Anwendung entworfen worden. Es kann alle mögliche Arten von Storage Typen (\acs{SAN}, \acs{NAS}, \acs{DAS}) anbinden und diese über eine einheitliche Schnittstelle dem Nutzer zur Verfügung stellen. Die verschiedene Hardware kann abgestuft werden, sodass alte oder kalte Daten auf langsame Speicher verschoben oder in die Cloud gespeichert werden.  Durch die \gls{POSIX} Konformität von \acs{GPFS} ist ein einfaches Navigieren sämtlicher Dateien im Cluster möglich und Veränderungen werden direkt angewendet %\parencite{snowflake.2016}.

Insgesamt gewinnt \acs{GPFS} im Sinne von Performance, da es ein Kernel-Level Dateisystem ist auf das wesentlich schneller zugegriffen werden kann. Es existieren Erweiterungen für Cloud Tiering und Sharing, die in dieser Form nicht bei HDFS existieren. Jede Art von Anwendung kann durch \acs{GPFS} profitieren, auch klassische Anwendungen, die nicht in einem verteilten System laufen.

Außerdem entfallen möglichen Kosten von Spectrum Scale, da diese Arbeit in einer IBM Abteilung angefertigt wird und somit einfacher Zugriff auf die Software besteht.

Aus den oben diskutierten Gründen wird in dieser Arbeit \textbf{IBM Spectrum Scale} verwendet.

\section{Auswahl des Cloud Anbieters}

Bei der Auswahl des Cloud Anbieters sind einige Punkte zu beachten. Zum einen sollten \acs{PaaS} Funktionalitäten gegeben sein, damit eine vollständige Anwendung entwickelt und hochgeladen werden kann und zum anderen sollte ein Service vorhanden sein, der als Cloud Speicher mit dem zuvor ausgewählten Spectrum Scale kombinierbar ist.
Cloud Tiering und Sharing sind einsetzbar mit allen S3 konformen Objekten Speichern %\parencite[S. 110]{ibm.2017}. 

Diese beiden Punkte sind bei allen drei Anbieter erfüllt.

Bei \acs{GPFS} kann zusätzlich noch ein IBM eigenes Format verwendet werden, um die Daten zwischen dem lokalen und Cloud System auszutauschen.
Von der Netzwerkgröße und Verfügbarkeit gewinnt \acs{AWS}, da es die meisten Server weltweit und auch speziell in Deutschland bereitstellt. Trotzdem sind ebenfalls Server von Microsoft und IBM in Deutschland vorhanden, sodass keine wesentliche Nachteile entstehen.

Microsoft Azure lässt sich in optimaler Form (ohne das Aufsetzen einer dedizierten \acs{VM}) mit .NET Laufzeiten benutzen, was die Auswahl der Entwicklungswerkzeuge wesentlich einschränkt.

Bluemix bietet einige DevOps Features an, die die Entwicklung massiv beschleunigen, das Aufsetzen einer Anwendung auf \acs{AWS} kann sich als komplizierter erweisen.

Zusätzlich entfallen sämtliche Kosten bei IBM Bluemix, da ein Firmen interner Account genutzt werden kann und auch zusätzliche Kosten für weitere verwendete Dienste entfallen.

Besonders aus letzterem Grund, obwohl AWS oder Azure nicht ungeeigneter sind, wird für die Entwicklung der Demo \textbf{IBM Bluemix} verwendet.

\section{Demoentwicklung - Auswahl der Laufzeitumgebung und Dienste}

Wie in \autoref{subsec:cloudprovider} beschrieben wurde, stellen \acs{PaaS} Umgebungen verschiedene Laufzeitumgebungen zur Verfügung. Zusätzlich wird eine Vielzahl von Diensten angeboten, um klassische Probleme wie Speicher, Kommunikation oder Analytics zu lösen.

An dieser Stelle wird kurz beschrieben, welche Produkte zur Erstellung der Demoapplikation verwendet werden. 

Als Entwicklersprache wird node.js Version 7.5.0 verwendet. Node ist eine auf der Chrome V8 basierende Javascript Engine. Sie verwendet ein nicht blockierendes I/O Modelle, dass besonders gut für parallel auftretende Netzwerkanfragen eingesetzt werden kann. In Kombination mit dieser Sprache wird der \ac{npm} benutzt \parencite{nodejs.2017}.

Zur Erweiterung der Serverfähigkeiten von Node.js wird express verwendet. Dies ist ein Framework zur Entwicklung von Webanwendung, dass eine Vielzahl nützlicher Middleware unterstützt.

Zur Versionsverwaltung wird \gls{Git} verwendet, auf welches auch die \ac{CD} Pipeline von Bluemix zugreift. Diese lädt den Quellcode aus dem Repository herunter, installiert die notwendigen Abhängigkeiten (Diese werden bei \acs{npm} in einer Paketbeschreibungsdatei hinterlegt: \textit{package.json}) und verschiebt die fertig gebaute Anwendung in das Cloud Foundry Server Netzwerk.  

Alle späteren Änderungen auf dem \textit{develop} Entwicklungszweig werden automatisch in Bluemix über dem oben beschrieben Weg angewendet. Über die grafische Oberfläche von Bluemix können zusätzliche Dienste konfiguriert werden.

Zur Kommunikation mit der S3 Schnittstelle des \acs{COS} wird ein Abstraktion der \acs{AWS} \gls{SDK} verwendet. Mithilfe von dieser kann mit allen \acs{S3} kompatiblen Diensten kommuniziert werden.

Des weiteren werden folgende, im Projekt wichtige, offene Bibliotheken verwendet:

\begin{table}
	\centering
		\begin{tabular}{ l | c | p{10cm}}
			\hline
			Paketname & Version & Beschreibung \\ \hline
			Node.js & 7.5.0 & Javascript Laufzeit Umgebung \\
			Express & 4.15 & Framework für Webanwendungen \\
			JQuery & 3.2.1 & Eine Sammlung von Werkzeugen um auf unterschiedlichen Browsern javascript Aufgaben zu vereinheitlichen \\
			Bootstrap & 3.3.7 & Grafik Framework, um responsive Webseiten zu designen. \\
			Moment & 2.18.1 & Bibliothek zum Umgang mit Zeitzonen und generellen Zeitberechnungen \\
			Filesize & 3.5.10 & Bibliothek zur automatischen Konvertierung von Dateigrößen \\
			aws-sdk & 2.87.0 & AWS SDK, wird verwendet um S3 kompatible Dienste anzusprechen \\
			ejs & 2.5.6 & HTML ähnliches Template zur Verwendung mit express \\
			body-parser & 1.17.2 & Express Middleware zum Parsen von HTTP Körpern in unterschiedlichen Formaten (z.B. JSON)\\
			Morgan & 1.8.2 & Logging Middleware für HTTP Anfragen in express \\
			Multer & 1.3.0 & Middleware zum Umgang mit Multipart/Datei Uploads von Clientanwendungen \\
			Toastr & 2.1.2 & Frontend Bibliothek zum Erzeugen von Toast-Nachrichten \\
			Watson Developer Cloud & 2.34.0 & Modul zum Zugriff auf verschiedenen Watson Dienste, in diesem Fall wird Ein Bildanalyse Tool eingebunden
		\end{tabular}
		\caption{Verwendete NPM-Pakete}
		\label{tab:npmpackages}
\end{table}


\todo{Update mit relevanten Tools, die später noch verwendet werden}    