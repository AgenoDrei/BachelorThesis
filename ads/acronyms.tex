%!TEX root = ../dokumentation.tex

\addchap{\langabkverz}
%nur verwendete Akronyme werden letztlich im Abkürzungsverzeichnis des Dokuments angezeigt
%Verwendung: 
%		\ac{Abk.}   --> fügt die Abkürzung ein, beim ersten Aufruf wird zusätzlich automatisch die ausgeschriebene Version davor eingefügt bzw. in einer Fußnote (hierfür muss in header.tex \usepackage[printonlyused,footnote]{acronym} stehen) dargestellt
%		\acs{Abk.}   -->  fügt die Abkürzung ein
%		\acf{Abk.}   --> fügt die Abkürzung UND die Erklärung ein
%		\acl{Abk.}   --> fügt nur die Erklärung ein
%		\acp{Abk.}  --> gibt Plural aus (angefügtes 's'); das zusätzliche 'p' funktioniert auch bei obigen Befehlen
%	siehe auch: http://golatex.de/wiki/%5Cacronym
%	
\begin{acronym}[YTMMM]
\setlength{\itemsep}{-\parsep}

\acro{API}{Application Programming Interface}
\acro{GB}{Gigabyte}
\acro{GFS}{Google File System}
\acro{HDFS}{Hadoop Distributed File System}
\acro{GPFS}{General Parallel File System}
\acro{HTTP}{Hypertext Transfer Protocol}
\acro{IDE}{Integrated Development Environment}
\acro{IP}{Internetprotokoll}
\acro{KB}{Kilobyte}
\acro{LTS}{Long Term Support}
\acro{MB}{Megabyte}
\acro{NAS}{Network Attached Storage}
\acro{DAS}{Direct Attached Storage}
\acro{NFS}{Network File System}
\acro{OS}{Operating System}
\acro{PDF}{Portable Document Format}
\acro{RSA}{Rivest, Shamir und Adleman}
\acro{SAN}{Storage Attached Network}
\acro{SSH}{Secure Shell}
\acro{SCP}{Secure Copy}
\acro{VM}{Virtuelle Maschine}
\acro{RPM}{Rotations per Minute}
\acro{SATA}{Serial Advanced Technology Attachment}
\acro{SAS}{Serial Attached \gls{SCSI}}
\acro{JBOD}{Just a bunch of disks}
\acro{NSD}{Network Shared Disks}
\acro{ILM}{Information Lifecycle Managment}
\acro{YARN}{Yet another Resource Negotiator}
\acro{WORM}{write-once-read-many}
\acro{UUID}{Universially Unique Identifier}
\acro{IoT}{Internet of Things}
\acro{PaaS}{Platform as a Service}
\acro{IaaS}{Infrastructure as a Service}
\acro{SaaS}{Software as a Service}
\acro{AWS}{Amazon Web Services}
\acro{S3}{Simple Storage Service}
\acro{VM}{Virtuelle Maschine}
\acro{CI}{Continious Integration}
\acro{CD}{Continious Delivery}
\acro{npm}{Node Package Manager}
\acro{COS}{IBM Cloud Object Storage}
\acro{WAN}{Wide Area Network}
\acro{KVM}{Kernel Virtual Machine}
\acro{QEMU}{Quick Emulator}
\acro{MVC}{Model View Controller}
\acro{UI}{Nutzerinterface}
\acro{BI}{Business Intelligence}
\acro{SSD}{Solid State Drive}
\end{acronym}
