%!TEX root = ../dokumentation.tex

%
% vorher in Konsole folgendes aufrufen:
%	makeglossaries makeglossaries dokumentation.acn && makeglossaries dokumentation.glo
%

%
% Glossareintraege --> referenz, name, beschreibung
% Aufruf mit \gls{...}
%

\newglossaryentry{Git}{name={Git},plural={Git},description={Git ist ein kostenloses System zur Versionskontrolle für kleine wie auch sehr große Projekte. ({\url{http://git-scm.com/}})}}

\newglossaryentry{Plugin}{name={Plugin},plural={Plugins},description={\flqq Zusatzprogramm, welches über eine vordefinierte Schnittstelle in ein Basisprogramm eingebunden wird und dessen Funktionsumfang erweitert. [...] [Stammen] oftmals von anderen Herstellern als das Basisprogramm. [...] Plug-ins sind oft aus eigenständigen Programmen entstanden und können deshalb [...] i.d.R. auch ohne das Basisprogramm verwendet werden\frqq\footcite[]{Lackes.2015}}}

\newglossaryentry{Bluemix}{name={IBM Bluemix},plural={Bluemix},description={Bluemix ist das Cloud Angebot von IBM. Es ist ein Platform as a Service Angebot, das die Entwicklung mithilfe von verschiedenen Sprachen und unterschiedlichen Diensten von IBM und Drittanbietern ermöglicht. Bluemix basiert auf der Infrastruktur von SoftLayer.}}

\newglossaryentry{IOPS}{name={IOPS},description={Input/Ouput Operations pro Sekunde. Wird verwendet, um die Schreib und Lesegeschwindigkeit von unterschiedlichen Speichergeräten zu vergleichen \url{https://en.wikipedia.org/wiki/IOPS}.}}

\newglossaryentry{SCSI}{name={SCSI},description={Eine Sammlung von Befehlen und Standards für die Verbindung und Kommunikation mit Peripheriegeräten \url{https://de.wikipedia.org/wiki/Small_Computer_System_Interface}.}}

\newglossaryentry{DevOps}{name={DevOps},description={Eine verkürzte Form von Entwicklung (Development) und Operationen (Operations) ist eine Software Entwicklungs- und Verteilungsprozess, der besonders Wert auf die Kommunikation zwischen diesen beiden, klassischerweise getrennten, Abteilungen legt. Hierdurch sollen massive Zeitvorteile bei der Erzeugung von neuen Produkten entstehen \url{https://en.wikipedia.org/wiki/DevOps}.}}

\newglossaryentry{POSIX}{name={POSIX},description={Eine Reihe von Standards, um die Kompatibilität zwischen Betriebssystemen zu gewährleisten. Es werden Schnittstellen, Kommandozeilen Befehle und Werkzeuge vorgeschrieben, um eine Verwendung zwischen unterschiedlichen Unix zu ermöglichen \url{https://en.wikipedia.org/wiki/POSIX}.}}

\newglossaryentry{SDK}{name={SDK},description={Ein Sofware Development Kit stellt Schnittstellen, Bibliotheken und Werkzeuge für Entwickler zur Verfügung, um auf diesen basierende Anwendungen zu erstellen \url{https://de.wikipedia.org/wiki/Software_Development_Kit}.}}

\newglossaryentry{On Premise}{name={On Premise},description={Mit On Premise bzw. Off Premise beschreibt man Software, die entweder auf dem Gelände des Kunden mit dessen eigener Hardware oder außerhalb von diesem (Cloud Computing) betrieben wird \url{https://de.wikipedia.org/wiki/On_Premises}.}}

\newglossaryentry{REST}{name={REST},description={Representational State Transfer bezeichnet ein Programmdesign für verteilte Systeme, das besonders bei Webanwendungen häufig verwendet wird. Es ist eine simple Alternative zu anderen Schnittstellen, da das Internet mit HTTP bereits einen Großteil der notwendigen Infrastruktur bereitstellt \url{https://en.wikipedia.org/wiki/Representational_state_transfer}.}}

\newglossaryentry{AJAX}{name={AJAX},description={Asynchronous Javascipt und XML beschreibt ein Konzept der asynchronen Dateiübertragung zwischen Browser und Servern. Es ermöglicht HTTP-Anfragen ausführen, auch nachdem eine Webseite bereits geladen wurde \url{https://en.wikipedia.org/wiki/Ajax_(programming)}.}}

\newglossaryentry{Promise}{name={Promise},description={Hierbei handelt es sich um ein Programmierkonstrukt, dass Synchrone Abläufe vereinfacht. Ein Promise ist ein Objekt, das als Proxy für ein noch nicht bekanntes oder vollständiges Ergebnis repräsentiert \url{https://en.wikipedia.org/wiki/Futures_and_promises}. }}