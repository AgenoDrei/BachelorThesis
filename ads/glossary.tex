%!TEX root = ../dokumentation.tex

%
% vorher in Konsole folgendes aufrufen:
%	makeglossaries makeglossaries dokumentation.acn && makeglossaries dokumentation.glo
%

%
% Glossareintraege --> referenz, name, beschreibung
% Aufruf mit \gls{...}
%

\newglossaryentry{Git}{name={Git},plural={Git},description={Git ist ein kostenloses System zur Versionskontrolle für kleine wie auch sehr große Projekte. ({\url{http://git-scm.com/}})}}

\newglossaryentry{Plugin}{name={Plugin},plural={Plugins},description={\flqq Zusatzprogramm, welches über eine vordefinierte Schnittstelle in ein Basisprogramm eingebunden wird und dessen Funktionsumfang erweitert. [...] [Stammen] oftmals von anderen Herstellern als das Basisprogramm. [...] Plug-ins sind oft aus eigenständigen Programmen entstanden und können deshalb [...] i.d.R. auch ohne das Basisprogramm verwendet werden\frqq\footcite[]{Lackes.2015}}}

\newglossaryentry{Bluemix}{name={IBM Bluemix},plural={Bluemix},description={Bluemix ist das Cloud Angebot von IBM. Es ist ein Platform as a Service Angebot, das die Entwicklung mithilfe von verschiedenen Sprachen und unterschiedlichen Diensten von IBM und Drittanbietern ermöglicht. Bluemix basiert auf der Infrastruktur von SoftLayer.}}

\newglossaryentry{IOPS}{name={IOPS},description={Input/Ouput Operations pro Sekunde. Wird verwendet, um die Schreib und Lesegeschwindigkeit von unterschiedlichen Speichergeräten zu vergleichen \url{https://en.wikipedia.org/wiki/IOPS}.}}

\newglossaryentry{SCSI}{name={SCSI},description={Eine Sammlung von Befehlen und Standards für die Verbindung und Kommunikation mit Peripheriegeräten \url{https://de.wikipedia.org/wiki/Small_Computer_System_Interface}.}}